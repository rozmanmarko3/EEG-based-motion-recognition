\documentclass[12pt,a4paper,titlepage,openany]{report}
\usepackage{style}

\usepackage[slovene]{babel}
\usepackage[utf8]{inputenc}
\usepackage[T1]{fontenc}
\usepackage{lmodern}
\usepackage{csquotes}
\usepackage{hyperref}

\usepackage[
backend=biber,
style=numeric,
sorting=ynt
]{biblatex}

\addbibresource{bibliografija.bib}


% Glava dokumenta:

\fancyhf{}
\lhead[]{{\fontsize{9.3}{12}\selectfont
Rozman M. Razpoznavanje gibanja na osnovi elektroencefalografije.\\
\noindent Univerza na Primorskem, Fakulteta za matematiko, naravoslovje in informacijske tehnologije, 2024}}
\chead[]{\fancyplain{}{}}
\rhead[]{\fancyplain{\thepage}
{\thepage}}
\cfoot[]{\fancyplain{}{}}
\lfoot[]{\fancyplain{}{}}
\rfoot[]{\fancyplain{}{}}
\normalsize

%%%%%%%%%%%%%%%%%%%%%%%%% ZAČETEK DOKUMENTA %%%%%%%%%%%%%%%%%%%%%%%%%%%%%%%%%%%%%%%%%%5

%%%%%%%%%%%%%%%%%%%%%%%%% Naslovna stran %%%%%%%%%%%%%%%%%%%%%%%%%


\begin{document}
\pagenumbering{Roman}
\pagestyle{empty}
\begin{center}
\noindent \large UNIVERZA NA PRIMORSKEM\\
\large FAKULTETA ZA MATEMATIKO, NARAVOSLOVJE IN\\
INFORMACIJSKE TEHNOLOGIJE


\normalsize
\vspace{6cm}
Zaključna naloga\\
\textbf{\large Razpoznavanje gibanja na osnovi elektroencefalografije}\\
\normalsize
(Movement recognition based on electroencephalography)\\
\end{center}

\begin{flushleft}
\vspace{5cm}
\noindent Ime in priimek: Marko Rozman
% v zgornjo vrstico dopišite ime in priimek študenta
\\
\noindent Študijski program: Računalništvo in informatika
% v zgornjo vrstico dopišite ime študijskega programa
\\
\noindent Mentor: doc. dr. Peter Rogelj 
% v zgornjo vrstico dopišite akademski naziv, ime in priimek mentorja

\end{flushleft}

\vspace{4cm}
\begin{center}
\large \textbf{Koper, Julij 2024}
% dopišite mesec in leto oddaje zaključne naloge
\end{center}
\newpage

\pagestyle{fancy}
%%%%%%%%%%%%%%%%%%%%%%%%%%%%%%% Ključna dokumentacijska informacija (slo in ang) %%%%%%%%%%%

\section*{Ključna dokumentacijska informacija}

\medskip
\begin{center}
\fbox{\parbox{\linewidth}{
\vspace{0.2cm}
\noindent
Ime in PRIIMEK: Marko ROZMAN\vspace{0.5cm}\\
Naslov zaključne naloge: Razpoznavanje gibanja na osnovi elektroencefalografije\vspace{0.5cm}\\
Kraj: Koper\vspace{0.5cm}\\
Leto: 2024\vspace{0.5cm}\\
Število listov: 34\hspace{2cm} Število slik: 15\hspace{2.6cm} Število tabel: 2\hspace{2cm}\vspace{0.5cm}\\
Število referenc: 14\vspace{0.5cm}\\
Mentor: doc. dr. Peter Rogelj\vspace{0.5cm}\\
Ključne besede: elektroencefalografija, Grangerjev index vzročnosti, kompleksni Pearsonov korelacijski koeficient, nevronska mreža, razvrščanje  \vspace{0.5cm}\\
{\bf Izvleček:}\\
Namen naloge je spoznati metode za razpoznavanje gibanja na osnovi elektroencefalografije. Gibanje smo razpoznavali iz podatkov EEG Motor Movement/Imagery Dataset in podatkov ki smo jih posneli sami na napravi Cognionics Quick-20. Iz posnetkov smo izračunali matrike povezljivosti z Grangerjevim indexom vzorčnosti in kompleksnim Pearsonovim korelacijskim koeficientom ki smo jih nato razvrstili z različnimi algorimi, vključno z nevronskimi mrežami. Naši rezultati kažejo da je v določenih primerih razvrščanje bolj točno z uporabo kompleksnega Pearsonovega korelacijskega koeficienta. 
\vspace{0.2cm}
}}
\end{center}

\newpage

\section*{Key words documentation}

\medskip

\begin{center}
\fbox{\parbox{\linewidth}{
\vspace{0.2cm}
\noindent
Name and SURNAME:\vspace{0.5cm}\\
Title of final project paper:\vspace{0.5cm}\\
Place:\vspace{0.5cm}\\
Year:\vspace{0.5cm}\\
Number of pages:\hspace{1.6cm} Number of figures:\hspace{2.2cm} Number of tables:\vspace{0.5cm}\\
Number of appendices:\hspace{0.6cm} Number of appendix pages:\hspace{0.8cm}Number of references:\vspace{0.5cm}\\
Mentor: title~First Name~Last Name, PhD\vspace{0.5cm}\\
% opomba: za "title" vpišite eno od naslednjega:
% Assist.~Prof. (če je naziv docent),
% Assoc.~Prof. (če je naziv izredni profesor),
% Prof. (če je naziv profesor)
Co-Mentor:\vspace{0.5cm}\\
Keywords:\vspace{0.5cm}\\
Math.~Subj.~Class.~(2010):\vspace{0.5cm}\\
{\bf Abstract:}
\vspace{0.2cm}
}}
\end{center}




%%%%%%%%%%%%%%%%%%%%%%%%%%%%%%% Zahvala %%%%%%%%%%%%%%%%%%%%%%%%%%%%%%%%%%%%%

\newpage
\section*{Zahvala}
Iskreno se zahvaljujem svojemu mentorju, doc. dr. Petru Roglju, za neprecenljivo podporo in vodenje pri pisanju diplomske naloge. Njegova strokovna pomoč pri izbiri metod, implementaciji ter pisanju je bila ključnega pomena na vsakem koraku. Hvaležen sem za priložnost dela s fizično napravo in za redne konzultacije ob sredah, ki so pripomogle k jasnosti in uspešnosti mojega dela.

Prav tako se iz srca zahvaljujem prijateljem in družini za njihovo neomajno podporo in spodbudo skozi celoten proces.

%%%%%%%%%%%%%%%%%%%%%%%%%%%%% Kazala %%%%%%%%%%%%%%%%%%%%%%%%%%%%%%
\newpage

% Dodamo kazala (po potrebi):
\tableofcontents
\addtocontents{toc}{\protect\thispagestyle{fancy}}
\newpage
\listoftables
\addtocontents{lot}{\protect\thispagestyle{fancy}}
\newpage
\listoffigures
\addtocontents{lof}{\protect\thispagestyle{fancy}}
\newpage
% ker priloge niso oštevilčene, tudi pikic do številk strani (ki jih ni) ne izpišemo
\renewcommand{\cftdot}{}
\listofappendices
\thispagestyle{fancy}
\newpage

\chapter*{Seznam kratic}
\thispagestyle{fancyplain}
\begin{longtable}{@{}p{1cm}@{}p{\dimexpr\textwidth-1cm\relax}@{}}

\nomenclature{$EEG$}{electroencephalography}
\nomenclature{$MMID$}{Motor Movement/Imagery Dataset} 
\nomenclature{$PLI$}{phase lag index}
\nomenclature{$wPLI$}{weighted phase lag index}
\nomenclature{$k-NN$}{k nearest neighbours}
\nomenclature{$SVM$}{support vector machine}
\nomenclature{$CPCC$}{complex Pearson correlation coefficient}
\nomenclature{$GC$}{Granger causality}

\end{longtable}
\newpage

\normalsize

%%%%%%%%%%%%%%%%%%%%%%%%%%%%%%%%%% Poglavja: %%%%%%%%%%%%%%%%%%%%%%%%%%%%%%%%%%%%%

% Namig: Za večjo preglednost datoteke lahko vsebino vsakega poglavja shranite v poseben .tex dokument
% v isto mapo, kjer je shranjena osnovna .tex datoteka. Nato poglavja vstavite v dokument s klicem \include
% Primer: PrvoPoglavje.tex in DrugoPoglavje.tex vstavimo tako:
% \include{PrvoPoglavje}
% \include{DrugoPoglavje}

\chapter{Uvod}
\thispagestyle{fancy}
\pagenumbering{arabic}
Motivacija za raziskavo je bilo ugotoviti do kakšne mere je mogoča razpoznavanje gibanja v živo na osnovi analize možganske aktivnosti z EEG meritvami. Najprej smo podatke iz prosto dostopne zbirke podatkov s pomočjo knjižnice EEGLAB razdelili na nekaj različno dolgih epoh po dogodkih in jim zožili frekvenčne pasove. Iz vsake pridobljene zbirke podatkov smo pridobili matrike povezljivosti Grangerjevega indexa vzročnosti in matrike povezljivosti kompleksnega Pearsonovega korelacijskega koeficienta. Na pridobljenih podatkih smo naučili nevronsko mrežo. Iz pridobljenih rezultatov smo se odločili za nadaljevanje razvoja na zbirki, ki je obetala najboljšo natančnost. Da bi omogočili delovanje v realnem času smo sami implementirali nekaj že obstoječih funkcij iz knjižnice. Posneli smo podatke na Cognionics Quick-20 in dodatno naučili nevronsko mrežo na naših podatkih za boljšo klasifikacijo.


\section{Elektroencefalografija}
Elektroencefalografija je metoda za merjenje možganske električne aktivnosti. Meri električne potenciale na površini temena ki jih deloma generira možganska aktivnost. V zadnjem stoletju so znanstveniki s pomočjo EEG pridobili vpogled v različne nevrološke bolezni. V zadnjem času pa se pojavlja interes v modeliranju eeg signalov in uporabo le teh za nadzor fizičnih naprav. EEG signali so običajno razdeljeni v območja ki odražajo različne spektralne vrhove. Ta območja so običajno določena kot delta (1-4 Hz), theta (4-8 Hz), alpha (8-13 Hz), beta (13-20 Hz), in gamma (<20 Hz). 
 \cite{nunez_electroencephalography_2016}

\subsection{Mednarodni sitem 10-20 pozicioniranja elektrod}
Mednarodni sistem 10-20 standardizira mesta elektrod tako, da so te  nameščene v mrežo od naziona do iniona ter od desnega do levega sluhovoda v presledkih 10 in 20 odstotkov razdalje. Vsaka elektroda je označena z črko lokacijo: T-Temporal, F- Frontal, P-Parietal, C-Central in O-Occipital, ter z črko z za elektrode na sredini glave, lihimi številkami za levo polovico glave in sodimi za desno. \cite{klem_ten-twenty_1999}

\section{Povezljivost}
Povezljivost se nananaša na vzorce nastale zaradi anatomskih povezav možganov, statistične odvisnosti ali interakcij med posameznimi deli možganov.  Enote med katerimi se meri povezljivost so lahko različne: posamezni nevroni, nevronske populacije, v našem primeru pa regije možganske skorje. Možganska aktivnost je omejena s povezljivostjo, le ta pa je zato ključnega pomena za razumevanje delovanja možganov. V grobem poznamo dve vrsti povezljivosti: strukturno in funkcijsko. Strukturna povezanost se nanaša na to kako so deli možganov med seboj fizično povezani. Funkcijska povezljivost pa se nanaša na to kako različni deli možganov med seboj komunicrajo oziroma sodelujejo.\cite{sporns_brain_2007}

\textcolor{red}{Mogoče tu bolj natančno o funkcijski povezljivosti, podnaslov?}


\chapter{Metode}

\section{Razvojno okolje}
Ves razvoj je potekal v programskem okolju MATLAB. Ta poleg samega programskega jezika vsebuje velik nabor že implementiranh funkcij, napredne aplikacije za strojno učenje in knjžnice ki omogočajo povezave z laboratorjskimi napravami. V njem sta ustvarjeni funkciji za računanje matric Grangerjevega indexa vzročnosti
in matric Kompleksnega Pearsonov korelacijskega koeficienta. V njem so ustvarjene nevronske mreže in uporabljeno je za ostale klasifikatorje. Prav tako smo v njem napisali funkcijo za zajemanje podatkov iz naprave Cognionics Quick-20, funkcijo ki v realnem času razpoznava gibanji.
    
\subsection{EEGLAB}
EEGLAB je interaktivna matlab orodjarna, za procesiranje in obdelavo elektrofizioloških podatkov. Omogoča rereferenciranje EEG signalov, izbiro določenih elektrod, deljenje podatkov na epohe glede na dogodke in filtriranje frekvenc. Omogoča interakcijo preko uporabniškega vmesnika. Vse akcije v vmesniku se prevedejo v ukaze ki jih lahko uporabimo v svoji kodi. Pri izdelavi naloge smo največ uporabljali funkcije branja .edf datotek, filtriranja frekvenc signalov in deljanja posnetkov na manjše dele.\cite{noauthor_eeglab_nodate}

\subsection{Lab streaming layer}
Lab streaming layer je odprtokodna vmesna programska oprema ki omogoča pošiljanje, prejemanje, sinhronizacijo in snemanje tokov podatkov. Omogoča enostavno povezovanje EEG naprave z programsko opremo MATLAB. Knjižnjico je potrebno prenesti in nato zgraditi na svojem računalniku.  \cite{noauthor_lsl-website_nodate}

\section{EEG Motor Movement/Imagery Dataset}
EEG Motor Movement/Imagery Dataset je prosto dostopna zbirka več kot 1500 eno in dve minutnih posnetkov 109 prostovoljcev ki opravljajo različne naloge. Za nas relevantni so posnetki serij 3, 5 in 7 v katerih prostovoljci stiskajo in sproščajo levo ali desno pest. Posnetki so shranjeni v formatu EDF+ ki vsebuje posnetke EEG in oznake dogodkov. Snemanje je bilo opravljeno s frekvenco 160Hz in 64 elektrodnim sistemom EEG.\cite{schalk_eeg_2009,schalk_bci2000_2004}

\section{Metode povezljivosti}
\subsection{Grangerjev index vzročnosti}
Grangerjev index vzročnosti je statistična metoda za preverjanje ali ena časovna vrsta nosi informacije o drugi. Metoda je bila razvita v šestdesedih letih devetnajstega stoletja za uporabo ekonomiji.

Za dve časovni vrsti $X_1$ in $X_2$, in $p$ kot število prejšnjih vrednosti ki jih upoštevamo pri računanju, lahko izračunamo $E_1$ in $E_1$ ki so napake pri predvidevanju naslednje vrednosti v vrsti $X_1$. V kolikor je varianca vrednosti $E_2$ manjša kot varianca vrednosti $E_1$ lahko predvidevamo da časovna vrsta $X_2$ nosi informacije o časovni vrsti $X_1$
\begin{align*}
X_1(t) &= \sum_{j=1}^{p} A_{1,j} X_1(t-j) + E_1(t)\\
X_1(t) &= \sum_{j=1}^{p} A_{2,j} X_1(t-j) + \sum_{j=1}^{p} A_{3,j} X_2(t-j) + E_2(t)
\end{align*}


\textcolor{red}{Mogoče razlaga kaj so A-ji. Ali so pravilno zapisani?}

\cite{seth_granger_2007}

\subsection{Kompleksni Pearsonov korelacijski koeficient}
Pearsonov korelacijski koeficient je najpogosteje uporabljen linearni korelacijski koeficient. Zanj smo se odločili saj v članku  \citetitle{sverko_complex_2022} avtorji pokažejo da vsebuje informacije PLI in wPLI ki sta dve najbolj pogosto uporabljeni metodi povezljivosti. Za naš primer lahko uporabimo enačbo s kompleksnimi števili:

\begin{align*}
r(X_1, X_2) &= \frac{\sum\limits_{n=1}^{N} X_{1,n} \cdot \overline{X_{2,n}}}{\sqrt{\sum\limits_{n=1}^{N} |X_{1,n}|^2} \cdot \sqrt{\sum\limits_{n=1}^{N} |X_{2,n}|^2}}.
\end{align*}
\textcolor{red}{Kaj predstavlja pika na koncu enačbe?}


\cite{sverko_complex_2022} 

\section{Klasifikacija}
Na pridobljenih matrikah povezljivosti smo preizkusili več vrst klasifikacije in sicer: odločitvena drevesa, metodo k najbližjih sosedov (k-NN), logistično regresijo, podporne vektorske stroje (SVM) in nevronske mreže.
\subsection{Classification learner}
Classification learner je aplikacija v Matlabu za enostavno klasifikacijo podatkov. Podpira različne metode klasifikacije, navzkrižno validacijo in uporabo različnih podatkov za gradnjo in testiranje modela.
\subsection{Nevronska mreža}
Nevronska mreža je sestavljena iz vhodne plasti za slike dimenzij 19x19x1, polno povezanega sloja s 100 nevroni, Leaky ReLU sloja, dropout sloja z 50\% verjetnostjo opustitve nevronov, polno povezanega sloja z 10 nevroni, GELU sloja, dropout sloja z 50\% verjetnostjo opustitve nevronov, polno povezanega sloja s tremi nevroni in Softmax sloja.
\begin{figure}[h!]
\begin{center}
\includegraphics[width=0.5\linewidth]{slike/Neural network.png}
\end{center}
\caption{Nevronska mreža.}
\end{figure}

\section{Filtriranje}
Knjižnica EEGLAB nam omogoča filtriranje signalov naprej in nazaj? kar v našem primeru ni primerno saj podatke prejemamo sekvenčno. Zato smo podatke filtrirali s pomočjo filtra z stanji ki nam omogoča filtriranje sekvenčnih podatkov. Ker pa filtra nista enakovredna, saj prvi ohranja zamike faz drugi pa ne, uporabljena metoda CPCC pa deluje na zamikih faz, smo izvedli dodatno testiranje da smo preverili če pristop deluje enako učinkovito.





\chapter{Rezultati}
\section{Delitev podatkov}
Za končno raziskavo smo imeli na voljo 7368 primerov stanj iz podatkovne zbirke EEG Motor Movement/Imagery Dataset. Primere stanj smo skrčili na enakomerno razporeditev, z 2456 primeri vsakega stanja. Sami smo posneli neka minut posnetkov, 186 primerov stanj od tega 62 primerov vsakega stanja. Za učenje nevronskih mrež smo uporabljali množice za učenje z 75\% podatkov in množice za testiranje z 25\% podatkov.

\section{Izbira metode povezljivosti}
Ker je kompleksni Pearsonov korelacijski koeficient izračunan iz analitičnih signalov ga lahko definiramo samo za ozke frekvenčne pasove. Pri računanju Grangerjevega indexa vzročnosti te omejitve ni, tako da smo ga lahko računali na celotnem frekvenčnem območju do 45Hz. Prav tako se je pojavilo vprašanje koliko dolgo epoho EEG signala bomo potrebovali za uspešno klasifikacijo. Kot možnosti smo vzeli prvo sekundo, prvi dve sekundi, drugi dve sekundi in prve štiri sekunde po dogodku. Točnost klasifikacije smo ocenili z zgoraj navedeno nevronsko mrežo. Za najboljšo metodo se je izkazal kompleksni Pearsonov korelacijski koeficient na območju 13-30Hz z najdalšimi epohami, 4s.
\begin{figure}[h!]
    \begin{center}
    \includegraphics[width=0.5\linewidth]{slike/Comparison.png}
    \end{center}
    \caption{Primerjava območij in dolžin epoh.}
\end{figure}

\section{Primerjava filtrov}
Knjižnica EEGLAB vsebuje samo filtre z ničelno fazo, ki filtrirajo naprej in nato nazaj po času, kar v našem primeru ni primerno saj podatke prejemamo sekvenčno, zato smo podatke filtrirali s pomočjo Butterworthovega filtra ki vsebuje stanja. Stanja nam omogočajo filtriranje sekvenčnih podatkov saj preprečijo napako na začetku filtra kjer le ta potrebuje predpostaviti začetno staje vseh signalov 0. Ker filtra nista enakovredna saj prvi ne spreminja faz drugi pa jih zamakne, uporabljena metoda CPCC pa deluje na zamikih faz, smo izvedli dodatno testiranje, da smo preverili če pristop deluje enako učinkovito.
\begin{figure}[h!]
    \begin{center}
    \includegraphics[width=0.5\linewidth]{slike/ComparisonFilters.png}
    \end{center}
    \caption{Primerjava klasifikacije CPCC z nevronsko mrežo za epoho 0-4s za različne frekvenčne pasove \textcolor{red}{ val $= eeglab - Butterworth$ testirano za 30 ljudi * 3 serije} }
\end{figure}

\begin{figure}[h!]
    \begin{center}
    \includegraphics[width=0.5\linewidth]{slike/Confusion_eeglab.png}
    \end{center}
    \caption{Matrika zmede nevronske mreže naučene na podatkih fitriranih s filtrom z ničelno fazo.}
    \end{figure}
    
    \begin{figure}[h!]
    \begin{center}
    \includegraphics[width=0.5\linewidth]{slike/Confusion_my.png}
    \end{center}
    \caption{Matrika zmede nevronske mreže naučene na podatkih fitriranih z Butterworthovim filtrom.}
    \end{figure}



\section{Rezultati na MMID}
\subsection{Classification learner}
Z uporabo aplikacije clasifiacation learner smo testirali več načinov klasifikacije in dosegl 49\% točnost. 
\begin{table}[h]
\centering
\begin{tabular}{|c|c|}
\hline
Metoda klasifikacije & točnost \\
\hline
odločitvena drevo & 40\%  \\
\hline
k-NN & 41\% \\
\hline
logistična regresija & 49\% \\
\hline
SVM & 45\% \\
\hline
\end{tabular}
\caption{Točnost klasifikacij}
\end{table}

\subsection{Nevronska mreža}
Nato smo poskusili z nevronsko mrežo ki je dosegla 52\% točnost.

\begin{figure}[h!]
\begin{center}
\includegraphics[width=0.5\linewidth]{slike/Confusion_13-20Hz_0s-4s.png}
\end{center}
\caption{Matrika zmede nevronske mreže naučene na podatkih zbirke.}
\end{figure}



\section{Rezultati na lastnih podatkih}
Da bi se približali pogojem v realnem času, smo nevronsko mrežo dodatno naučili na naših podatkih. Zaradi različnih pogojev snemanja in đnčnosti naprav na katerih so podatki snemani je točnost klasifikacije pričakovano padla.
\begin{figure}[h!]
\begin{center}
\includegraphics[width=0.5\linewidth]{slike/Confusion_13-20Hz_0s-4s_retrained.png}
\end{center}
\caption{Matrika zmede nevronske mreže dodatno naučene na naših podatkih.}
\end{figure}
\section{Preizkus v realnem času}






\chapter{Zaključki}
V nalogi smo uspešno razpoznali gibanje iz EEG signalov, tako iz podatkovne zbirke MMID, posnete po mednarodnem sistemu 10-10, kot tudi iz podatkov, posnetih na napravi Cognionics Quick-20. Posnete signale smo obdelali z različnimi pristopi. Signale smo rerefernecirali, filtrirali s filtrom z ničelno fazo in Butterworthovim filtrom na običajna območja zanimanja pri analizi EEG signalov. Nato smo signale razdelili na različno dolge epohe in izbrali najustreznejše. Obdelane signale smo pretvorili v matrike povezljivosti s pomočjo Grangerjevega indeksa vzročnosti in kompleksnega Pearsonovega korelacijskega koeficienta. Pridobljene matrike smo razvrstili z aplikacijo Clasification Learner in z nevronsko mrežo, ki smo jo implementirali sami. Dosegli smo zadovoljive točnosti na podatkih MMID in podatkih, posnetih z napravo Cognionics Quick-20. Metode, ki smo jih uporabljali, omogočajo nadaljnjo analizo, ki lahko prispeva k boljšemu razumevanju možganskih aktivnosti kot direktnega razvrščanja signalov. Z uporabo kompleksnega Pearsonovega korelacijskega koeficienta smo pokazali, da je razpoznavanje gibanja mogoče iz krajših epoh območja beta. Ugotovili smo, da kompleksni Pearsonov korelacijski koeficient zagotavlja boljšo metodo za izračun povezljivosti kot tradicionalno uporabljeni Grangerjev indeks vzročnosti. Za delo v realnem času smo sami implementirali in ocenili primernost filtrov, ki jih knjižnica EEGLAB ne podpira. Pokazali smo, da je najustreznejše sprotno razvrščanje z metodo povezljivosti kompleksnega Pearsonovega korelacijskega koeficienta.\\
Glavne omejitve, ki nam onemogočajo še točnejše razvrščanje z uporabljenimi metodami,  so omejena velikost posnetkov in omejena natančnost naprav EEG. Prav tako naloga vsebuje omejitev pri učenju klasifikatorjev, saj sistema nismo preizkusili pri razvrščanju EEG signalov oseb, na čigar signalih klasifikator ni bil naučen. Točnost sprotnega razvrščanja signalov ene osebe pa je dodatno omejena z njenim razpoloženjem.\\
Razpoznavanje gibanja iz signalov EEG ima potenciale aplikacije v medicini, zlasti pri razvoju sistemov za nadzor protez in rehabilitacijskih naprav. Metode povezljivosti, uporabljene v nalogi, pa nam lahko poglobijo razumevanje možganske aktivnosti med različnimi fizičnimi nalogami.



\printbibliography

\addtocontents{toc}{\setcounter{tocdepth}{2}}
\end{document}
