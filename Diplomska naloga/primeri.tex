\documentclass[12pt,a4paper,titlepage,openany]{report}
\usepackage{style}

% Glava dokumenta:

\fancyhf{}
\lhead[]{{\fontsize{9.3}{12}\selectfont
Priimek I. Naslov zaključne naloge.\\
\noindent Univerza na Primorskem, Fakulteta za matematiko, naravoslovje in informacijske tehnologije, leto}}
\chead[]{\fancyplain{}{}}
\rhead[]{\fancyplain{\thepage}
{\thepage}}
\cfoot[]{\fancyplain{}{}}
\lfoot[]{\fancyplain{}{}}
\rfoot[]{\fancyplain{}{}}
\normalsize

%%%%%%%%%%%%%%%%%%%%%%%%% ZAČETEK DOKUMENTA %%%%%%%%%%%%%%%%%%%%%%%%%%%%%%%%%%%%%%%%%%5

%%%%%%%%%%%%%%%%%%%%%%%%% Naslovna stran %%%%%%%%%%%%%%%%%%%%%%%%%


\begin{document}
\pagenumbering{Roman}
\pagestyle{empty}
\begin{center}
\noindent \large UNIVERZA NA PRIMORSKEM\\
\large FAKULTETA ZA MATEMATIKO, NARAVOSLOVJE IN\\
INFORMACIJSKE TEHNOLOGIJE


\normalsize
\vspace{6cm}
Zaključna naloga\\
\textbf{\large Naslov zaključne naloge}\\
\normalsize
(Naslov zaključne naloge v angleškem jeziku)\\
\end{center}

\begin{flushleft}
\vspace{5cm}
\noindent Ime in priimek:
% v zgornjo vrstico dopišite ime in priimek študenta
\\
\noindent Študijski program:
% v zgornjo vrstico dopišite ime študijskega programa
\\
\noindent Mentor:
% v zgornjo vrstico dopišite akademski naziv, ime in priimek mentorja
\\
\noindent Somentor:
% če imate somentorja, v zgornjo vrstico dopišite akademski naziv, ime in priimek somentorja
% če somentorja nimate, zbrišite zgornjo in spodnjo vrstico
\\
\end{flushleft}

\vspace{4cm}
\begin{center}
\large \textbf{Koper, mesec leto}
% dopišite mesec in leto oddaje zaključne naloge
\end{center}
\newpage

\pagestyle{fancy}
%%%%%%%%%%%%%%%%%%%%%%%%%%%%%%% Ključna dokumentacijska informacija (slo in ang) %%%%%%%%%%%

\section*{Ključna dokumentacijska informacija}

\medskip
\begin{center}
\fbox{\parbox{\linewidth}{
\vspace{0.2cm}
\noindent
Ime in PRIIMEK:\vspace{0.5cm}\\
Naslov zaključne naloge:\vspace{0.5cm}\\
Kraj:\vspace{0.5cm}\\
Leto:\vspace{0.5cm}\\
Število listov: \hspace{2cm} Število slik: \hspace{2.6cm} Število tabel:\hspace{2cm}\vspace{0.5cm}\\
Število prilog: \hspace{1.9cm} Število strani prilog: \hspace{1cm} Število referenc:\vspace{0.5cm}\\
Mentor:\vspace{0.5cm}\\
Somentor:\vspace{0.5cm}\\
Ključne besede:\vspace{0.5cm}\\
Math.~Subj.~Class.~(2010):\vspace{0.5cm}\\
{\bf Izvleček:}\\
Izvleček predstavlja kratek, a jedrnat prikaz vsebine naloge. V največ 250 besedah nakažemo problem, metode, rezultate, ključne ugotovitve in njihov pomen.
\vspace{0.2cm}
}}
\end{center}

\newpage

\section*{Key words documentation}

\medskip

\begin{center}
\fbox{\parbox{\linewidth}{
\vspace{0.2cm}
\noindent
Name and SURNAME:\vspace{0.5cm}\\
Title of final project paper:\vspace{0.5cm}\\
Place:\vspace{0.5cm}\\
Year:\vspace{0.5cm}\\
Number of pages:\hspace{1.6cm} Number of figures:\hspace{2.2cm} Number of tables:\vspace{0.5cm}\\
Number of appendices:\hspace{0.6cm} Number of appendix pages:\hspace{0.8cm}Number of references:\vspace{0.5cm}\\
Mentor: title~First Name~Last Name, PhD\vspace{0.5cm}\\
% opomba: za "title" vpišite eno od naslednjega:
% Assist.~Prof. (če je naziv docent),
% Assoc.~Prof. (če je naziv izredni profesor),
% Prof. (če je naziv profesor)
Co-Mentor:\vspace{0.5cm}\\
Keywords:\vspace{0.5cm}\\
Math.~Subj.~Class.~(2010):\vspace{0.5cm}\\
{\bf Abstract:}
\vspace{0.2cm}
}}
\end{center}




%%%%%%%%%%%%%%%%%%%%%%%%%%%%%%% Zahvala %%%%%%%%%%%%%%%%%%%%%%%%%%%%%%%%%%%%%

\newpage
\section*{Zahvala}


Tu se zahvalimo sodelujočim pri zaključni nalogi, osebam ali ustanovam, ki so nam pri delu pomagale ali so delo omogočile. Zahvalimo se lahko tudi mentorju in morebitnemu somentorju.

%%%%%%%%%%%%%%%%%%%%%%%%%%%%% Kazala %%%%%%%%%%%%%%%%%%%%%%%%%%%%%%
\newpage

% Dodamo kazala (po potrebi):
\tableofcontents
\addtocontents{toc}{\protect\thispagestyle{fancy}}
\newpage
\listoftables
\addtocontents{lot}{\protect\thispagestyle{fancy}}
\newpage
\listoffigures
\addtocontents{lof}{\protect\thispagestyle{fancy}}
\newpage
% ker priloge niso oštevilčene, tudi pikic do številk strani (ki jih ni) ne izpišemo
\renewcommand{\cftdot}{}
\listofappendices
\thispagestyle{fancy}
\newpage

\chapter*{Seznam kratic}
\thispagestyle{fancyplain}
\begin{longtable}{@{}p{1cm}@{}p{\dimexpr\textwidth-1cm\relax}@{}}
\nomenclature{$tj.$}{to je}
\nomenclature{$npr.$}{na primer}
\end{longtable}
\newpage

\normalsize

%%%%%%%%%%%%%%%%%%%%%%%%%%%%%%%%%% Poglavja: %%%%%%%%%%%%%%%%%%%%%%%%%%%%%%%%%%%%%

% Namig: Za večjo preglednost datoteke lahko vsebino vsakega poglavja shranite v poseben .tex dokument
% v isto mapo, kjer je shranjena osnovna .tex datoteka. Nato poglavja vstavite v dokument s klicem \include
% Primer: PrvoPoglavje.tex in DrugoPoglavje.tex vstavimo tako:
% \include{PrvoPoglavje}
% \include{DrugoPoglavje}

\chapter{Uvod}
\thispagestyle{fancy}
\pagenumbering{arabic}

Tu opišemo problem, ki ga v zaključni nalogi obravnavamo. Predstavimo osnovne ideje in uvedemo
osnovne definicije in oznake. V uvodu lahko tudi povzamemo matematična dejstva, ki jih bomo
kasneje uporabili. Citiramo literaturo, ki je relevantna za obravnavane pojme, lahko tudi dodatno literaturo.


\medskip
\noindent Zgled citiranja:

\medskip
Minimiziranje pasovnosti matrik pomaga pri njihovem shranjevanju in pri računanju z njimi, npr.~pri Gaussovi eliminaciji. Bralec bo podrobnosti našel v~\cite{Chinn, George, Strang}.


\chapter{Vložitve}
\thispagestyle{fancy}

Dodamo vezno besedilo.

\section{Široke vložitve}\label{siroke}

Dodamo vezno besedilo.

\subsection{Podpoglavje poglavja Široke vložitve}

Dodamo vezno besedilo.

\begin{defi}
Graf $G$ je {\em povezan}, če za vsaki dve točki $u,v\in V(G)$ obstaja vsaj ena $u$-$v$ pot v $G$.
\end{defi}

\begin{lema}
Lema je pomožna trditev, ki služi za dokaz glavnega izreka.
\end{lema}

\begin{proof}
Tu napišimo dokaz leme. Dokaz naj bo čim krajši, vendar razumljiv vsem študentom. Pazite na logično strukturo dokaza: Vsi koraki naj bodo utemeljeni.
\end{proof}

\begin{izr}
Izrek je najpomemnejša trditev v poglavju. Izrekov naj bo čim manj, preostale trditve formuliramo kot leme ali kot trditve.
\end{izr}

\begin{proof}
Tu napišemo dokaz izreka.
\end{proof}

\begin{posl}
Posledica je ugotovitev, ki neposredno sledi iz glavnega izreka. Potrebuje le krajši dokaz (par vrstic). Če se ne da dokazati v par vrsticah, potem to ni več posledica, temveč lema ali trditev.
\end{posl}

\begin{prim}
Z zgledom osvetlimo lemo ali glavni izrek. Zgled je lahko protiprimer k veljavnosti izreka, če mu izpustimo kakšno od hipotez.
\end{prim}

\newpage
Takole se vstavlja slika:

\bigskip
\bigskip
\begin{figure}[h!]
\begin{center}
\includegraphics[width=0.5\linewidth]{graf.pdf}
\end{center}
\caption{Vhodni podatki.}\label{slika:podatki}
\end{figure}

\newpage
 Takole se vstavlja tabela.

\begin{table}[h!]
\caption{Algoritem PLOGBAND}
\label{tabela:algoritem}
\fbox{%
      \parbox{\linewidth}{%
{\noindent \bf Algoritem PLOGBAND:}
\vskip 5pt
\indent Podatka: {\obeylines \indent \indent graf $G = G(V,E)$ na $n$ vozliščih in z $m$ povezavami,
% \indent \indent nenegativno celo število $L$.}
\indent \indent $L \in \N$.}
\begin{enumerate}

\item Za $1 \le j \le L$ naj bodo $p_j$ približno enakomerno
(geometrijsko) razporejena števila med $1 - 1/\log\log n$ in $1/\log n$.
Tj., vsa razmerja $p_j/p_{j+1}$ naj bodo približno enaka. (Natančne
formule so v~razdelku \ref{siroke}, kjer je opisana vložitev naključnih podmnožic.)

\item Uredi vozlišča glede na naraščajoče vrednosti $h(v)$. Vozlišča z enakimi
vrednostmi $h$ uredi poljubno.

\item Vrni urejeni seznam vozlišč kot linearno ureditev.
\end{enumerate}
}}
\end{table}

\noindent Takole navedemo sliko ali tabelo:

\medskip
Vse, kar potrebujemo za konec dokaza, je povzeto v Tabeli~\ref{tabela:algoritem}.

\medskip
Za primer vhodih podatkov glej Sliko~\ref{slika:podatki}.

\medskip
\noindent Podobno lahko označimo in navajamo razdelke, poglavja, izreke, ipd.


\newpage
Takole lahko zapišemo psevdokodo algoritma:

\begin{algorithm}[h!]\label{algoritem1}
\Vhod{Realni matriki $A$ in $B$ velikosti $n\times n$.}
\Izhod{Matrika $C = A\cdot B$.}
\caption{Množenje matrik}
{
    \Za{$i = 1, \ldots, n$}
    {
        \Za{$j = 1, \ldots, n$}
        {
            $C[i,j]:= A[i,1]\cdot B[1,j];$

            \Za{$k = 2, \ldots, n$}
            {
                $C[i,j]:= C[i,j]+A[i,k]\cdot B[k,j]$;
            }
        }
    }
    \Ce{$n = 2$}
    {
       ne naredi nič
    }
}
\Vrni{$C$;}
\end{algorithm}


\chapter{Naslov poglavja}
\thispagestyle{fancy}

Takole citiramo spletne vire:~\cite{splet1,splet2,splet3}.\\
Takole citiramo članke, sprejete v objavo ali v tisku:~\cite{Novak,Novak2,Novak3,Novak4}.\\
Takole citiramo članke, poslane v objavo:~\cite{Novak5,Novak6}.

%%%%%%%%%%%%%%%%%%%%%%%%%%%%%%%%%% Zaključek %%%%%%%%%%%%%%%%%%%%%%%%%%%%%%%%%%%%%
\chapter{Zaključek}
\thispagestyle{fancy}

V nekaj stavkih na kratko povzamemo, kaj smo v nalogi obravnavali.
Po želji lahko navedemo še kakšne dodatne reference za bralca, ki bi ga zanimalo kaj več, ipd.


%%%%%%%%%%%%%%%%%%%%%%%%%%%%%%%% Literatura %%%%%%%%%%%%%%%%%%%%%%%%%%%%%%%%%

 \begin{thebibliography}{99}
\thispagestyle{fancy}

\bibitem{Blum}
  \clanekVRevijiVecAvtorjev
    {A.~Blum, G.~Konjevod}{R.~Ravi}
    {Semidefinite relaxations for minimum bandwidth and other vertex-ordering problems}
   {Theor.~Comp.~Sci.}{235}
   {2000}{25--42}

\bibitem{Bourgain}
  \clanekVRevijiEnAvtor
    {J.~Bourgain}
    {On Lipschitz embedding of finite metric spaces in Hilbert space}
   {Israel J.~Math}{52}
   {1985}{46--52}

\bibitem{Chinn}
  \clanekVRevijiVecAvtorjev
    {P.~Chinn, J.~Chv\'atalov\'a, A.~Dewdney}{N.~Gibbs}
    {The bandwidth problem for graphs and matrices -- a survey}
   {J.~Graph Theory}{6}
   {1982}{223--254}

\bibitem{Chvatalova}
\doktorskaDisertacija
    {J.~Chv\'atalov\'a}
    {On the bandwidth problem for graphs}
    {Ph.D.~dissertation, University of Waterloo, 1980}

\bibitem{Frankl}
  \clanekVRevijiVecAvtorjev
    {P.~Frankl}{H.~Maehara}
    {The Johnson-Lindenstrauss lemma and the sphericity of some graphs}
   {J.~Comb.~Theory, Ser.~B}{44}
   {1988}{355--362}

\bibitem{Feige}
  \clanekVRevijiEnAvtor
    {U.~Feige}
    {Approximating the bandwidth via volume respecting embeddings}
    {J.~Comp.~Syst.~Sci.}{60}
    {2000}{510--539}

\bibitem{George}
  \knjigaVecAvtorjev {A.~George}{J.~Liu}
   {Computer Solution of Large Positive Definite Systems}
    {Prentice-Hall, 1981}

\bibitem{Grotschel}
  \knjigaVecAvtorjev  {M.~Gr\"otschel, L.~Lov\'asz}{A.~Schrijver}
   {Geometric Algorithms and Combinatorial Optimization}
    {Springer-Verlag, Berlin, 1987}

\bibitem{Kleitman}
   \clanekVRevijiVecAvtorjev
     {D.~Kleitman}{R.~Vohra}
     {Computing the bandwidth of interval graphs}
     {SIAM J.~Discrete Math.}{3}
     {1990}{373--375}

\bibitem{Knuth}
  \knjigaEnAvtor     {D.~Knuth}
   {The Art of Computer Programming, Vol.~2, Seminumerical Algorithms}
    {Addison Wesley, Second Edition, 1981}

\bibitem{Lagarias}
\poglavjeVKnjigiEnAvtor
   {J.~Lagarias}
   {Point Lattices}
   {R.~Graham, M.~Gr\"otschel, L.~Lov\'asz (ur.)}
   {Handbook of Combinatorics, Volume~1}
   {MIT Press, 1995}
   {919--966}

\bibitem{Linial}
   \clanekVRevijiVecAvtorjev
     {N.~Linial, E.~London}{Y.~Rabinovich}
   {The geometry of graphs and some of its algorithmic applications}
     {Combinatorica}{15}
     {1995}{215--245}

\bibitem{Novak}
   \clanekVRevijiEnAvtorSprejetVObjavo
     {J.~Novak}
   {Polynomial approximation of rational manifolds. I}
     {J.~Abstract~Approximation}

\bibitem{Novak2}
   \clanekVRevijiEnAvtorVTisku
     {J.~Novak}
   {Polynomial approximation of rational manifolds. II}
     {J.~Abstract~Approximation}

\bibitem{Novak3}
   \clanekVRevijiVecAvtorjevSprejetVObjavo
     {J.~Novak}{M.~Novak}
   {Polynomial approximation of rational manifolds. III}
     {J.~Abstract~Approximation}

\bibitem{Novak4}
   \clanekVRevijiVecAvtorjevVTisku
     {J.~Novak}{M.~Novak}
   {Polynomial approximation of rational manifolds. IV}
     {J.~Abstract~Approximation}

\bibitem{Novak5}
   \clanekVRevijiEnAvtorPoslanVObjavo
     {J.~Novak}
   {Polynomial approximation of rational manifolds. V}
     {2014}

\bibitem{Novak6}
   \clanekVRevijiVecAvtorjevPoslanVObjavo
     {J.~Novak}{M.~Novak}
   {Polynomial approximation of rational manifolds. VI}
     {2014}

\bibitem{Santalo}
  \knjigaEnAvtor     {L.A.~Santalo}
   {Integral Geometry and Geometric Probability}
    {Encyclopedia of Mathematics and its Applications, Volume 1, Addison Wesley, 1976}

\bibitem{Saxe}
   \clanekVRevijiEnAvtor
     {J.~Saxe}
   {Dynamic programming algorithms for recognizing small-bandwidth graphs in polynomial time}
     {SIAM J.~Alg.~Meth.}{1}
     {1980}{363--369}

\bibitem{Strang}
  \knjigaEnAvtor     {G.~Strang}
   {Linear Algebra and its Applications, Third Edition}
    {Saunders College \hbox{Publishing}, Harcourt Brace Jovanovich College Publishers, 1988}

\bibitem{Unger}
\konferencniClanekEnAvtor
    {W.~Unger}
    {The complexity of the approximation of the bandwidth problem}
    {Proc.~39th Annual IEEE Symposium on Foundations of Computer Science}
    {1998}
    {82--91}

\bibitem{splet1}
\spletniVirBrezAvtorja
    {Miller--Rabin primality test}
    {\newline http://en.wikipedia.org/wiki/Miller/\%E2\%80\%93Rabin\_primality\_test}
    {25}{4}{2014}

\bibitem{splet2}
\spletniVirBrezAvtorjaZInstitucijo
    {The Converse of Wilson's Theorem}{The Oxford Math Center}
    {http://www.oxfordmathcenter.com/drupal7/node/382}
    {25}{4}{2014}

\bibitem{splet3}
\spletniVirZAvtorjem
    {T.~Tao}
    {Algebraic probability spaces}
    {http://terrytao.wordpress.com/}
    {4}{7}{2014}

% Ena vrstica mora biti tu prazna zaradi pravilnih navedb na strani, kjer so reference citirane.
\end{thebibliography}
\newpage

%%%%%%%%%%%%%%%%%%%%%%%%%%%%%%%%%%%% Priloge %%%%%%%%%%%%%%%%%%%%%%%%%%%%%%%%%%%%%
\pagestyle{fancyplain}
\vspace*{\fill}
     \begin{center}
          \bf{\Huge{Priloge}}
     \end{center}
\vspace*{\fill}
\thispagestyle{fancy}

\appendix
\thispagestyle{empty}
\pagenumbering{gobble}

\addtocontents{toc}{\setcounter{tocdepth}{-1}}
\appendices{A Naslov prve priloge}
\chapter{Naslov prve priloge}
\thispagestyle{empty}
Tu dodamo prvo prilogo.

% pozor:
% ukaz
% \thispagestyle{empty}
% mora biti prisoten na vsaki strani priloge (da se ne prikaže glava dokumenta)

\appendices{B Naslov druge priloge}
\chapter{Naslov druge priloge}
\thispagestyle{empty}
Tu dodamo drugo prilogo.

% Pozor:
% ukaz
% \thispagestyle{empty}
% mora biti prisoten na vsaki strani priloge (da se ne prikaže glava dokumenta)

\addtocontents{toc}{\setcounter{tocdepth}{2}}
\end{document}
