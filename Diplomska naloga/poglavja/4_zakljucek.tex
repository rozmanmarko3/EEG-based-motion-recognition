\chapter{Zaključki}
V nalogi smo uspešno razpoznali gibanje iz EEG signalov, tako iz podatkovne zbirke MMID, posnete po mednarodnem sistemu 10-10, kot tudi iz podatkov, posnetih na napravi Cognionics Quick-20. Posnete signale smo obdelali z različnimi pristopi. Signale smo rerefernecirali, filtrirali s filtrom z ničelno fazo in Butterworthovim filtrom na običajna območja zanimanja pri analizi EEG signalov. Nato smo signale razdelili na različno dolge epohe in izbrali najustreznejše. Obdelane signale smo pretvorili v matrike povezljivosti s pomočjo Grangerjevega indeksa vzročnosti in kompleksnega Pearsonovega korelacijskega koeficienta. Pridobljene matrike smo razvrstili z aplikacijo Clasification Learner in z nevronsko mrežo, ki smo jo implementirali sami. Dosegli smo zadovoljive točnosti na podatkih MMID in podatkih, posnetih z napravo Cognionics Quick-20. Metode, ki smo jih uporabljali, omogočajo nadaljnjo analizo, ki lahko prispeva k boljšemu razumevanju možganskih aktivnosti kot direktnega razvrščanja signalov. Z uporabo kompleksnega Pearsonovega korelacijskega koeficienta smo pokazali, da je razpoznavanje gibanja mogoče iz krajših epoh območja beta. Ugotovili smo, da kompleksni Pearsonov korelacijski koeficient zagotavlja boljšo metodo za izračun povezljivosti kot tradicionalno uporabljeni Grangerjev indeks vzročnosti. Za delo v realnem času smo sami implementirali in ocenili primernost filtrov, ki jih knjižnica EEGLAB ne podpira. Pokazali smo, da je najustreznejše sprotno razvrščanje z metodo povezljivosti kompleksnega Pearsonovega korelacijskega koeficienta.\\
Glavne omejitve, ki nam onemogočajo še točnejše razvrščanje z uporabljenimi metodami,  so omejena velikost posnetkov in omejena natančnost naprav EEG. Prav tako naloga vsebuje omejitev pri učenju klasifikatorjev, saj sistema nismo preizkusili pri razvrščanju EEG signalov oseb, na čigar signalih klasifikator ni bil naučen. Točnost sprotnega razvrščanja signalov ene osebe pa je dodatno omejena z njenim razpoloženjem.\\
Razpoznavanje gibanja iz signalov EEG ima potenciale aplikacije v medicini, zlasti pri razvoju sistemov za nadzor protez in rehabilitacijskih naprav. Metode povezljivosti, uporabljene v nalogi, pa nam lahko poglobijo razumevanje možganske aktivnosti med različnimi fizičnimi nalogami.

