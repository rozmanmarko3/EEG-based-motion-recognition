\chapter{Zaključki}
V nalogi smo uspešno razpoznali gibanje iz EEG signalov. Uspešno smo razpoznali gibanje iz podatkovne zbirke MMID posnete po mednarodnem sistemu 10-10 in iz podatkov posnetih na napravi Cognionics Quick-20. Posnete signale smo obdelali z različnimi pristopi. Signale smo rerefernecirali, filtrirali z filtrom z ničelno fazo in butterworthovim filtrom na običanja območja zanimanja pri analizi EEG signalov. Nato smo signale razdelili na različno dolge epohe in izbrali najustreznejše. Obdelane signale smo pretvorili v matrice povezljivosti s pomočjo Granjgerjevega indexa vzorčnosti in kompleksnega Pearsonovega korelacijskega koeficienta. Pridobljene matrice smo klasificirali z aplikacijo Clasification Learner in z nevronsko mrežo ki smo jo implementirali sami. Dosegli smo zadovoljive natančnosti na podatkih MMID in podatkih posnetih z napravo Cognionics Quick-20. Metode ki smo jih uporabljali omogočajo boljše razumevanje možganskih aktivnosti kot direktna klasifikacija signalov, z uporabo kompleksnega Pearsonovega korelacijskega koeficienta smo demostrirali da je razpoznavanje gibanja mogoče iz krajših epoh območja beta. Ugotovili smo, da kompleksni Pearsonov korelacijski koeficient zagotavlja boljšo metodo za izračun povezljivosti kot tradicionalno uporabljeni Grangerjev index vzročnosti. Za delo v realnem času smo sami implementirali in ocenili primernost filtrov ki jih knjižnica EEGLAB ne podpira.\\
Glavne omejitve ki nam onemogočajo natančenjšo klasifikacijo z uporabljenimi metodami so omejena velikost posnetkov in omejena natnčnost naprav EEG. Prav tako naloga vsebuje omejitev pri učenju klasifikatorjev, saj sitema nismo preizkusili pri klasifikacijah EEG signalov oseb na čigar signalih kalsifikator ni bil učen.\\
Klasifikacija gibanja iz signalov EEG ima potenciale aplikacije v medicini, zlasti pri razvoju sitemov za nadzor protez inrehabilitacijskih naprav. Metode povezljivosti, uporabljene v nalogi pa nam lahko poglobijo razumevanje možganske aktivnosti med različnimi fizičnimi nalogami.